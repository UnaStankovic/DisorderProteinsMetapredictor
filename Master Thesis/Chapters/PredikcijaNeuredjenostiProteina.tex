% Chapter Template

\chapter{Predikcija neuređenosti proteina} % Main chapter title
\label{predikcija} % Change X to a consecutive number; for referencing this chapter elsewhere, use \ref{ChapterX}

Kao što je navedeno u prethodnom poglavlju, neuređenost proteina se može, osim eksperimentalnog, određivati i računarski. Upravo o tom vidu određivanja, odnosno, predikcije, neuređenosti proteina govori ovo poglavlje. Najpre, biće detaljnije opisan računarski postupak. Nakon toga, biće priče o prediktorima, od kojih će pojedini biti detaljnije objašnjeni. Na kraju, ukratko, će biti predstavljena baza podataka DisProt i njen značaj u ovom radu.\\\\



%----------------------------------------------------------------------------------------
%	SECTION 1
%----------------------------------------------------------------------------------------

\section{Prediktori}

%-----------------------------------
%	SUBSECTION 1
%-----------------------------------
\subsection{SPINE-D/ SPOT-D}


%-----------------------------------
%	SUBSECTION 2
%-----------------------------------

\subsection{PONDR}

\subsection{s2D}
\subsection{IUPred}
\subsection{ESpritz}
\subsection{SEG}
\subsection{Disopred2}

%----------------------------------------------------------------------------------------
%	SECTION 2
%----------------------------------------------------------------------------------------

\section{Baza podataka DisProt}

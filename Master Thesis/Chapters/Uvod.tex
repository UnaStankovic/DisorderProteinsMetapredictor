% Chapter 1

\chapter{Uvod} % Main chapter title

\label{Chapter1} % For referencing the chapter elsewhere, use \ref{Chapter1} 

%This is only a temporary version of introduction based on an application submitted to the committee

Proteini su biološki makromolekuli neophodni za izgradnju i pravilno funkcionisanje ćelija, i igraju mnogobrojne uloge u različitim procesima koji se odvijaju unutar organizma. Struktura proteina zavisi od rasporeda aminokiselina i utiče na njegovu funckiju. Primarna struktura podrazumeva niz aminokiselina koje učestvuju u izgradnji proteina, dok se sekundarna odnosi na oblik koji protein zauzima u prostoru (spirala ili traka). Proteine sa nestabilnom sekundarnom strukturom nazivamo neuređenim. Pored značajne uloge u obavljanju brojnih bioloških funkcija, otkriveno je i da postoji veza između ovih proteina i razvoja savremenih, neizlečivih bolesti i zbog toga su oni u fokusu bioinformatičke zajednice.\\\\

Neuređenost proteina se utvrđuje eksperimentalno, laboratorijskim analizama, ili uz pomoć prediktora za automatsko predviđanje neuređenosti proteina. Laboratorijske analize spadaju u spore, veoma skupe metode, koje ne mogu da odgovore na potrebe akademske zajednice i industrije. Iz tog razloga, poslednjih godina, došlo je do razvoja velikog broja alata za automatsko predviđanje neuređenosti proteina. Zbog velike brojnosti ovih alata, razvijaju se metaprediktori koji predstavljaju njihove kombinacije. Specifičan cilj ovog master rada je razvoj jednog metaprediktora za određivanje neuređenosti proteina koji bi konsenzusom objedinio rezultate najnovijih prediktivnih alata na osnovu metodologije na kojoj su zasnovani. Alat će biti testiran na skupu proteina sa eksperimentalno utvrđenom neuređenošću DisProt (eng.~{\em DisProt}).
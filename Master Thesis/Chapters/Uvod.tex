% Chapter 1

\chapter*{Uvod} % Main chapter title

\label{Chapter1} % For referencing the chapter elsewhere, use \ref{Chapter1} 

%This is only a temporary version of introduction based on an application submitted to the committee

Proteini su biološki makromolekuli neophodni za izgradnju i pravilno funkcionisanje ćelija i igraju mnogobrojne uloge u različitim procesima koji se odvijaju unutar organizma. Struktura proteina zavisi od redosleda aminokiselina i utiče na njegovu funckiju. Primarna struktura podrazumeva niz aminokiselina koje učestvuju u izgradnji proteina, dok se sekundarna odnosi na oblik koji protein zauzima u prostoru (spirala ili traka). Proteine sa nestabilnom sekundarnom strukturom nazivamo neuređenim. Pored značajne uloge u obavljanju brojnih bioloških funkcija, otkriveno je i postojanje veze između ovih proteina i razvoja neizlečivih bolesti i zbog toga su oni u fokusu bioinformatičke zajednice.\\

S obzirom na to da je priroda ovog rada multidisciplinarna, odnosno da pripada oblasti bioinformatike, neophodno je dati prigodan uvod koji bi čitaocu približio materiju. U prvom poglavlju date su biološke osnove bez kojih razumevanje motivacije, cilja i samog rada ne bi bilo moguće. Najpre je opisan protein, njegova struktura, značaj, funkcija i uloga u organizmu. Potom, poseban akcenat je stavljen na moguće strukture koje protein zauzima u prostoru i njihov izgled i uticaj. Na samom kraju poglavlja govori se o neuređenosti proteina iz biološkog ugla, kao uvodu u naredno poglavlje. \\

U drugom poglavlju, govori se o predikciji neuređenosti proteina iz ugla računarstva. Navedeno je nekoliko poznatijih prediktora, od kojih su neki  korišćeni pri razvoju metaprediktora, kao i o bazama podataka DisProt i UniProt. Uloga DisProt baze, u ovom radu, leži pri testiranju preciznosti predikcije nad rezultatima koje su vratili prediktori, dok je značaj UniProt baze u prikupljanju niski potrebnih za rad prediktora.\\

Naredno poglavlje govori o samom metaprediktoru. Aplikacija koja je razvijena se posmatra iz više uglova, to su ugao arhitekture i organizacije, funkcionalnosti i iz ugla korisnika. Poslednje poglavlje opisuje postignute rezultate i moguća unapređenja.




%Neuređenost proteina se utvrđuje eksperimentalno, laboratorijskim analizama, ili uz pomoć prediktora za automatsko predviđanje neuređenosti proteina. Laboratorijske analize spadaju u spore, veoma skupe metode, koje ne mogu da odgovore na potrebe akademske zajednice i industrije. Iz tog razloga, poslednjih godina, došlo je do razvoja velikog broja alata za automatsko predviđanje neuređenosti proteina. Zbog velike brojnosti ovih alata, razvijaju se metaprediktori koji predstavljaju njihove kombinacije. Specifičan cilj ovog master rada je razvoj jednog metaprediktora za određivanje neuređenosti proteina koji bi konsenzusom objedinio rezultate najnovijih prediktivnih alata na osnovu metodologije na kojoj su zasnovani. Alat će biti testiran na skupu proteina sa eksperimentalno utvrđenom neuređenošću DisProt (eng.~{\em DisProt}).